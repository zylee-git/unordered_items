% Homework template for Inference and Information
% UPDATE: September 26, 2017 by Xiangxiang
\documentclass[a4paper]{article}
\usepackage{ctex}
\usepackage{amsmath, amssymb, amsthm}
\usepackage{moreenum}
\usepackage{mathtools}
\usepackage{url}
\usepackage{bm}
\usepackage{enumitem}
\usepackage{graphicx}
\usepackage{listings}
\usepackage{multirow}
\usepackage{siunitx}
\lstset{
    basicstyle          =   \sffamily,          % 基本代码风格
    keywordstyle        =   \bfseries,          % 关键字风格
    commentstyle        =   \rmfamily\itshape,  % 注释的风格,斜体
    stringstyle         =   \ttfamily,  % 字符串风格
    flexiblecolumns,                % 
    numbers             =   left,   % 行号的位置在左边
    showspaces          =   false,  % 是否显示空格,显示了有点乱,所以不显示了
    numberstyle         =   \zihao{-5}\ttfamily,    % 行号的样式,小五号,tt等宽字体
    showstringspaces    =   false,
    captionpos          =   t,      % 这段代码的名字所呈现的位置,t指的是top上面
    frame               =   lrtb,   % 显示边框
}

\lstdefinestyle{Python}{
    language        =   Python, % 语言选Python
    basicstyle      =   \zihao{-5}\ttfamily,
    numberstyle     =   \zihao{-5}\ttfamily,
    keywordstyle    =   \color{blue},
    keywordstyle    =   [2] \color{teal},
    stringstyle     =   \color{magenta},
    commentstyle    =   \color{red}\ttfamily,
    breaklines      =   true,   % 自动换行,建议不要写太长的行
    columns         =   fixed,  % 如果不加这一句,字间距就不固定,很丑,必须加
    basewidth       =   0.5em,
}
% \usepackage{subcaption}
\usepackage[caption=false,font=footnotesize,labelfont=rm,textfont=rm,subrefformat=parens]{subfig}
\usepackage{booktabs} % toprule
\usepackage[mathcal]{eucal}
\usepackage{color}
\usepackage{iidef}
\newif\ifans\anstrue
\newcommand{\myspace}[1]{\par\vspace{#1\baselineskip}}

\thecourseinstitute{\textnormal{通信与网络}}
\theterm{确定}
\begin{document}



\vspace{3mm}
\centerline{\textbf{\Large{实验4$\quad$差错控制编码实验报告}}}

\setcounter{section}{4}

\section{实验内容}
\subsection{(7,4)汉明码的纠错实验}

\begin{enumerate}[label=(\arabic*)]
    \item 记录不同信道误符号率$\varepsilon$下,有汉明码编译码和无汉明码编译码时的误块率和误比特率。
\begin{table}[htbp]
\centering
\begin{tabular}{cc|c|c|c|c|c|c}
\hline
\multicolumn{2}{c|}{信道误符号率}                          & 0.001     & 0.005 & 0.01 & 0.05& 0.1&0.2\\ \hline
\multicolumn{1}{c|}{\multirow{3}{*}{无汉明码编译码}} & 数据块数 &   &     &     &  & &   \\ \cline{2-8} 
\multicolumn{1}{c|}{}                       & 误比特率  &   &     &     &   & &  \\ \cline{2-8}
\multicolumn{1}{c|}{}                       & 误块率  &   &     &     &   & &  \\ \hline
\multicolumn{1}{c|}{\multirow{3}{*}{有汉明码编译码}} & 数据块数 &   &     &     &  & &   \\ \cline{2-8} 
\multicolumn{1}{c|}{}                       & 误比特率  &   &     &     &   & &  \\ \cline{2-8}
\multicolumn{1}{c|}{}                       & 误块率  &   &     &     &   & &  \\ \hline
\end{tabular}
    \caption{差错控制编码实验记录}
    \label{tab:laplace_junyun}
\end{table}
    \item 给出误码率随信道误符号率变化的曲线图。
    \item 分析曲线变化原因。


\end{enumerate}


\subsection{(7,4)汉明码的交织实验}

\begin{enumerate}[label=(\arabic*)]
    \item 调试交织函数$interleaver$和解交织函数$deinterleaver$
    \begin{enumerate}
    \item 给出使用[1,2,3,...,35]的整数序列时,交织前后及解交织前后的序列。
    \item 给出使用全零的整数序列,突发错误长度为交织块行数时,解交织前后的错误图案。
    \item 给出使用全零的整数序列,突发错误长度为交织块行数的2倍时,解交织前后的错误图案。
    \end{enumerate}
    \item 差错控制编码与交织
    \begin{enumerate}
    \item 给出所使用的生成矩阵$\mathbf{G}$。
    \item 给出原始的信息序列$x$,汉明码编码后的序$x\_code$,交织后的序列$x\_interleave$,经过信道传输后的序列$y$,解交织后的序列$y\_deinterleave$,以及纠错后的序列$y\_decode$。
    \item 分析交织的作用和效果。
    \item 记录不同信道突发错误长度下,有无交织/解交织时的误块率和误比特率。
    \begin{table}[htbp]
\centering
\begin{tabular}{cc|c|c|c|c|c|c}
\hline
\multicolumn{2}{c|}{信道突发错误长度}                          & 3     & 5 & 10 & 15& 20&25\\ \hline
\multicolumn{1}{c|}{\multirow{3}{*}{无交织/解交织}} & 数据块数 &   &     &     &  & &   \\ \cline{2-8} 
\multicolumn{1}{c|}{}                       & 误比特率  &   &     &     &   & &  \\ \cline{2-8}
\multicolumn{1}{c|}{}                       & 误块率  &   &     &     &   & &  \\ \hline
\multicolumn{1}{c|}{\multirow{3}{*}{有交织/解交织}} & 数据块数 &   &     &     &  & &   \\ \cline{2-8} 
\multicolumn{1}{c|}{}                       & 误比特率  &   &     &     &   & &  \\ \cline{2-8}
\multicolumn{1}{c|}{}                       & 误块率  &   &     &     &   & &  \\ \hline
\end{tabular}
    \caption{差错控制编码实验记录}
    \label{tab:laplace_junyun}
\end{table}
    \end{enumerate}
\end{enumerate}

\section{实验总结、体会和建议}

\end{document}




%%% Local Variables:
%%% mode: late\rvx
%%% TeX-master: t
%%% End:
