% Homework template for Inference and Information
% UPDATE: September 26, 2017 by Xiangxiang
\documentclass[a4paper]{article}
\usepackage{ctex}
\usepackage{amsmath, amssymb, amsthm}
\usepackage{moreenum}
\usepackage{mathtools}
\usepackage{url}
\usepackage{bm}
\usepackage{enumitem}
\usepackage{graphicx}
\usepackage{listings}
\usepackage{multirow}
\usepackage{siunitx}
\lstset{
    basicstyle          =   \sffamily,          % 基本代码风格
    keywordstyle        =   \bfseries,          % 关键字风格
    commentstyle        =   \rmfamily\itshape,  % 注释的风格,斜体
    stringstyle         =   \ttfamily,  % 字符串风格
    flexiblecolumns,                % 
    numbers             =   left,   % 行号的位置在左边
    showspaces          =   false,  % 是否显示空格,显示了有点乱,所以不显示了
    numberstyle         =   \zihao{-5}\ttfamily,    % 行号的样式,小五号,tt等宽字体
    showstringspaces    =   false,
    captionpos          =   t,      % 这段代码的名字所呈现的位置,t指的是top上面
    frame               =   lrtb,   % 显示边框
}

\lstdefinestyle{Python}{
    language        =   Python, % 语言选Python
    basicstyle      =   \zihao{-5}\ttfamily,
    numberstyle     =   \zihao{-5}\ttfamily,
    keywordstyle    =   \color{blue},
    keywordstyle    =   [2] \color{teal},
    stringstyle     =   \color{magenta},
    commentstyle    =   \color{red}\ttfamily,
    breaklines      =   true,   % 自动换行,建议不要写太长的行
    columns         =   fixed,  % 如果不加这一句,字间距就不固定,很丑,必须加
    basewidth       =   0.5em,
}
% \usepackage{subcaption}
\usepackage[caption=false,font=footnotesize,labelfont=rm,textfont=rm,subrefformat=parens]{subfig}
\usepackage{booktabs} % toprule
\usepackage{adjustbox}
\usepackage[mathcal]{eucal}
\usepackage{color}
\usepackage{iidef}
\newif\ifans\anstrue
\newcommand{\myspace}[1]{\par\vspace{#1\baselineskip}}

\thecourseinstitute{\textnormal{通信与网络}}
\theterm{确定}
\begin{document}



\vspace{3mm}
\centerline{\textbf{\Large{实验4$\quad$差错控制编码实验报告}}}

\setcounter{section}{4}

\section{实验内容}
\subsection{(7,4)汉明码的纠错实验}

\begin{enumerate}[label=(\arabic*)]
    \item 记录不同信道误符号率$\varepsilon$下,有汉明码编译码和无汉明码编译码时的误块率和误比特率。
\begin{table}[htbp]
\centering
\begin{adjustbox}{max width=\textwidth}
\begin{tabular}{cc|c|c|c|c|c|c}
\hline
\multicolumn{2}{c|}{信道误符号率}                          & 0.001     & 0.005 & 0.01 & 0.05& 0.1&0.2\\ \hline
\multicolumn{1}{c|}{\multirow{3}{*}{无汉明码编译码}} & 数据块数 & 1000000  &   200000  &  100000   & 20000 &10000 &  10000 \\ \cline{2-8} 
\multicolumn{1}{c|}{}                       & 误比特率  & 0.0010  &  0.0050   &   0.0010  & 0.0504 & 0.1004& 0.2007 \\ \cline{2-8}
\multicolumn{1}{c|}{}                       & 误块率  & 0.0041  &  0.0197   &   0.0394  &  0.1865 & 0.3454& 0.5954 \\ \hline
\multicolumn{1}{c|}{\multirow{3}{*}{有汉明码编译码}} & 数据块数 &  1000000 &   200000  &   100000  & 20000 & 10000& 10000  \\ \cline{2-8} 
\multicolumn{1}{c|}{}                       & 误比特率  & $7.5\times 10^{-6}$  &  $2.16\times 10^{-4}$   & $9.1\times 10^{-4}$  & 0.0200 &  0.0682 &0.1967   \\ \cline{2-8}
\multicolumn{1}{c|}{}                       & 误块率  & $1.9\times 10^{-5}$  &  $4.95\times 10^{-4}$   &  0.0022& 0.0458  &  0.1554 & 0.4270  \\ \hline
\end{tabular}
\end{adjustbox}
    \caption{差错控制编码实验记录}
    \label{tab:laplace_junyun}
\end{table}
    \item 给出误码率随信道误符号率变化的曲线图。\par
    见图 \ref{fig:1_3}.
    \begin{figure}[htbp]
        \centering
        \includegraphics[scale=0.2]{img/1_3.jpg}
        \caption{误码率随信道误符号率变化曲线图}
        \label{fig:1_3}
    \end{figure}
    \item 分析曲线变化原因。\par
    在低信道误符号率时,有编码的误比特率明显低于无编码的情况,这是因为汉明码能够有效纠正单个随机错误。\par
    在高信道误符号率时,有编码的误比特率可能接近甚至高于无编码的情况,这是因为当错误位数超过1位时,汉明码可能产生错误的纠错,导致更多错误,这体现了汉明码只能纠正单个错误的局限性。


\end{enumerate}


\subsection{(7,4)汉明码的交织实验}

\begin{enumerate}[label=(\arabic*)]
    \item 调试交织函数$interleaver$和解交织函数$deinterleaver$
    \begin{enumerate}
    \item 给出使用[1,2,3,...,35]的整数序列时,交织前后及解交织前后的序列。\par
    交织前序列: 1 2 3 4 5 6 7 8 9 10 11 12 13 14 15 16 17 18 19 20 21 22 23 24 25 26 27 28 29 30 31 32 33 34 35 \par
    交织后序列: 1 8 15 22 29 2 9 16 23 30 3 10 17 24 31 4 11 18 25 32 5 12 19 26 33 6 13 20 27 34 7 14 21 28 35 \par
    解交织后序列: 1 2 3 4 5 6 7 8 9 10 11 12 13 14 15 16 17 18 19 20 21 22 23 24 25 26 27 28 29 30 31 32 33 34 35 \par
    \item 给出使用全零的整数序列,突发错误长度为交织块行数时,解交织前后的错误图案。\par
    解交织后序列: 0 0 0 0 0 0 0 0 0 1 0 0 0 0 0 0 1 0 0 0 0 0 0 0 0 0 0 0 0 1 0 0 0 0 0 \par
    \item 给出使用全零的整数序列,突发错误长度为交织块行数的2倍时,解交织前后的错误图案。\par
    解交织后序列: 0 0 1 1 0 0 0 0 0 0 1 0 0 0 0 0 1 0 0 0 0 0 0 0 0 0 0 0 0 1 1 0 0 0 0 
    \end{enumerate}
    \item 差错控制编码与交织
    \begin{enumerate}
    \item 给出所使用的生成矩阵$\mathbf{G}$。
    生成矩阵$\mathbf{G}$如下:
    \begin{equation}
        \mathbf{G}=
        \begin{bmatrix}
            1 & 0 & 0 & 0 & 1 & 1 & 0 \\
            0 & 1 & 0 & 0 & 1 & 0 & 1 \\
            0 & 0 & 1 & 0 & 0 & 1 & 1 \\
            0 & 0 & 0 & 1 & 1 & 1 & 1 \\
        \end{bmatrix}
    \end{equation}
    \item 给出原始的信息序列$x$,汉明码编码后的序$x\_code$,交织后的序列$x\_interleave$,经过信道传输后的序列$y$,解交织后的序列$y\_deinterleave$,以及纠错后的序列$y\_decode$。\par
    $x$: 1 1 0 1 1 0 0 1 1 1 0 1 1 0 1 0 0 1 1 1 \par
    $x\_code$: 1 1 0 1 1 0 0 1 0 0 1 0 0 1 1 1 0 1 1 0 0 1 0 1 0 1 0 1 0 1 1 1 0 0 1 \par
    $x\_interleave$: 1 1 1 1 0 1 0 1 0 1 0 0 0 1 1 1 1 1 0 1 1 0 1 1 0 0 0 0 0 0 0 1 0 1 1 \par
    $y$: 1 1 1 1 0 1 0 1 0 1 0 0 0 1 1 1 1 1 0 1 0 0 1 1 0 0 0 0 0 0 0 1 0 1 1 \par
    $y\_deinterleave$: 1 1 0 1 0 0 0 1 0 0 1 0 0 1 1 1 0 1 1 0 0 1 0 1 0 1 0 1 0 1 1 1 0 0 1 \par
    $y\_decode$: 1 1 0 1 1 0 0 1 1 1 0 1 1 0 1 0 0 1 1 1 
    \item 分析交织的作用和效果。
    交织的作用是使一个突发错误产生的错误比特分散至多个块中,从而尽量使每个块中的误比特数不多于 1,从而发挥汉明码的纠错作用,降低误比特率。\par
    效果是大大降低了最终的误比特率,甚至可将误比特率清零。
    \item 记录不同信道突发错误长度下,有无交织/解交织时的误块率和误比特率。
    \begin{table}[htbp]
\centering
\begin{adjustbox}{max width=\textwidth}
\begin{tabular}{cc|c|c|c|c|c|c}
\hline
\multicolumn{2}{c|}{信道突发错误长度}                          & 3     & 5 & 10 & 15& 20&25\\ \hline
\multicolumn{1}{c|}{\multirow{3}{*}{无交织/解交织}} & 数据块数 &  10000 &   10000  &  10000   &10000  & 10000& 10000  \\ \cline{2-8} 
\multicolumn{1}{c|}{}                       & 误比特率  &  0.0443 &  0.0790   &  0.1499   &  0.2190 & 0.2909& 0.3670 \\ \cline{2-8}
\multicolumn{1}{c|}{}                       & 误块率  &  0.0885 &  0.1551   &  0.2907   & 0.4226  &0.5544 & 0.6996 \\ \hline
\multicolumn{1}{c|}{\multirow{3}{*}{有交织/解交织}} & 数据块数 &  10000 &   10000  &   10000  &10000  & 10000&  10000 \\ \cline{2-8} 
\multicolumn{1}{c|}{}                       & 误比特率  &  0 &  0   &   0.1254  &  0.2505 & 0.3473& 0.4087 \\ \cline{2-8}
\multicolumn{1}{c|}{}                       & 误块率  &  0 &  0   &  0.2474   &  0.4960 & 0.6884& 0.8166 \\ \hline
\end{tabular}
\end{adjustbox}
    \caption{差错控制编码实验记录}
    \label{tab:laplace_junyun}
\end{table}
    \end{enumerate}
\end{enumerate}

\section{实验总结、体会和建议}
通过本次差错控制编码实验,我深入理解了汉明码的基本原理与性能特点,掌握了(7,4)汉明码的编码、译码及纠错过程。实验结果表明,在低误码率信道环境下,汉明码能够有效纠正单个随机错误,显著降低系统误比特率和误块率,体现了信道编码在提升通信可靠性方面的重要价值。然而当信道误符号率较高时,由于汉明码最小汉明距离为3,仅能纠正单错而无法应对多错,甚至可能因错误传播导致性能恶化,这让我深刻认识到编码方案的选择需与实际信道条件相匹配。在交织实验中,我进一步体会到交织技术通过时间维度上的分散处理,能够将突发错误转化为随机错误,从而充分发挥汉明码的纠错能力,这种“时间分集”的思想为解决无线通信中的深度衰落问题提供了有效途径。整个实验过程不仅锻炼了我的MATLAB编程与仿真能力,更让我对差错控制编码的理论与实践有了系统性的认识,理解了在实际通信系统中如何通过编码与交织的联合设计来平衡可靠性、效率与复杂度,为后续深入学习现代通信技术奠定了坚实基础。
\end{document}




%%% Local Variables:
%%% mode: late\rvx
%%% TeX-master: t
%%% End:
