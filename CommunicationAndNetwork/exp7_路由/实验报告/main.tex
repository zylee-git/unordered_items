% Homework template for Inference and Information
% UPDATE: January 3, 2026 by 李正扬
\documentclass[a4paper]{article}
\usepackage{ctex}
\usepackage{amsmath, amssymb, amsthm}
\usepackage{moreenum}
\usepackage{mathtools}
\usepackage{url}
\usepackage{bm}
\usepackage{enumitem}
\usepackage{graphicx}
\usepackage{listings}
\usepackage{multirow}
\usepackage{siunitx}
\lstset{
    basicstyle          =   \sffamily,          % 基本代码风格
    keywordstyle        =   \bfseries,          % 关键字风格
    commentstyle        =   \rmfamily\itshape,  % 注释的风格,斜体
    stringstyle         =   \ttfamily,  % 字符串风格
    flexiblecolumns,                % 
    numbers             =   left,   % 行号的位置在左边
    showspaces          =   false,  % 是否显示空格,显示了有点乱,所以不显示了
    numberstyle         =   \zihao{-5}\ttfamily,    % 行号的样式,小五号,tt等宽字体
    showstringspaces    =   false,
    captionpos          =   t,      % 这段代码的名字所呈现的位置,t指的是top上面
    frame               =   lrtb,   % 显示边框
}

\lstdefinestyle{Python}{
    language        =   Python, % 语言选Python
    basicstyle      =   \zihao{-5}\ttfamily,
    numberstyle     =   \zihao{-5}\ttfamily,
    keywordstyle    =   \color{blue},
    keywordstyle    =   [2] \color{teal},
    stringstyle     =   \color{magenta},
    commentstyle    =   \color{red}\ttfamily,
    breaklines      =   true,   % 自动换行,建议不要写太长的行
    columns         =   fixed,  % 如果不加这一句,字间距就不固定,很丑,必须加
    basewidth       =   0.5em,
}
% \usepackage{subcaption}
\usepackage[caption=false,font=footnotesize,labelfont=rm,textfont=rm,subrefformat=parens]{subfig}
\usepackage{booktabs} % toprule
\usepackage[mathcal]{eucal}
\usepackage{color}
\usepackage{iidef}
\newif\ifans\anstrue
\newcommand{\myspace}[1]{\par\vspace{#1\baselineskip}}

\thecourseinstitute{\textnormal{通信与网络}}
\theterm{确定}
\begin{document}



\vspace{3mm}
\centerline{\textbf{\Large{实验7$\quad$路由实验报告}}}

\setcounter{section}{4}

\section{实验内容}
\subsection{网络拓扑}
\subsection{RIP协议与DV算法仿真实验}
\begin{enumerate}
    \item 打开实验文件“exp7 1.m”,阅读实验代码并运行实验文件,记录Matlab命令行输出的不同迭代轮次下的距离矩阵和下一跳矩阵。选择感兴趣某一次迭代,结合代码对Bellman-Ford算法的实现,对距离矩阵和下一跳矩阵的更新给出解释。\par
    --------------- LOOP START --------------\par 
    距离矩阵
    \begin{equation}
        \begin{bmatrix}
            0  &  2  &  5  &  1 & \infty & \infty & \infty \\
            2  &  0  &  3  &  2 & \infty & \infty & \infty \\
            5  &  3  &  0  &  3  &  1  &  5 & \infty \\
            1  &  2  &  3  &  0  &  1 & \infty & \infty \\
          \infty & \infty  &  1  &  1  &  0  &  2 & \infty \\
          \infty & \infty  &  5 & \infty  &  2  &  0  &  1 \\
          \infty & \infty & \infty & \infty & \infty  &  1  &  0 \\        
        \end{bmatrix}
    \end{equation}\par
    下一跳矩阵
    \begin{equation}
        \begin{bmatrix}
            0  &  2  &  3  &  4  &  0  &  0  &  0 \\
            1  &  0  &  3  &  4  &  0  &  0  &  0 \\
            1  &  2  &  0  &  4  &  5  &  6  &  0 \\
            1  &  2  &  3  &  0  &  5  &  0  &  0 \\ 
            0  &  0  &  3  &  4  &  0  &  6  &  0 \\
            0  &  0  &  3  &  0  &  5  &  0  &  7 \\
            0  &  0  &  0  &  0  &  0  &  6  &  0 \\       
        \end{bmatrix}
    \end{equation}\par
    ---------------- LOOP 1 ----------------\par
    距离矩阵
    \begin{equation}
        \begin{bmatrix}
            0  &  2  &  4  &  1  &  2  &  9 & \infty \\
            2  &  0  &  3  &  2  &  3  &  8 & \infty \\
            3  &  3  &  0  &  2  &  1  &  3  &  6 \\
            1  &  2  &  2  &  0  &  1  &  3 & \infty \\
            2  &  3  &  1  &  1  &  0  &  2  &  3 \\
            4  &  5  &  3  &  3  &  2  &  0  &  1 \\
            5  &  6  &  4  &  4  &  3  &  1  &  0 \\
        \end{bmatrix}
    \end{equation}\par
    下一跳矩阵
    \begin{equation}
        \begin{bmatrix}
            0  &  2  &  4  &  4  &  4  &  4  &  0 \\
            1  &  0  &  3  &  4  &  4  &  3  &  0 \\
            5  &  2  &  0  &  5  &  5  &  5  &  6 \\
            1  &  2  &  5  &  0  &  5  &  5  &  0 \\
            4  &  4  &  3  &  4  &  0  &  6  &  6 \\
            5  &  5  &  5  &  5  &  5  &  0  &  7 \\
            6  &  6  &  6  &  6  &  6  &  6  &  0 \\       
        \end{bmatrix}
    \end{equation}\par
    ---------------- LOOP 2 ----------------\par
    距离矩阵
    \begin{equation}
        \begin{bmatrix}
            0  &  2  &  3  &  1  &  2  &  4  &  10 \\
            2  &  0  &  3  &  2  &  3  &  5  &  9 \\
            3  &  3  &  0  &  2  &  1  &  3  &  4 \\
            1  &  2  &  2  &  0  &  1  &  3  &  4 \\
            2  &  3  &  1  &  1  &  0  &  2  &  3 \\
            4  &  5  &  3  &  3  &  2  &  0  &  1 \\
            5  &  6  &  4  &  4  &  3  &  1  &  0 \\       
        \end{bmatrix}
    \end{equation}\par
    下一跳矩阵
    \begin{equation}
        \begin{bmatrix}
            0  &  2  &  4  &  4  &  4  &  4  &  4 \\
            1  &  0  &  3  &  4  &  4  &  4  &  3 \\
            5  &  2  &  0  &  5  &  5  &  5  &  5 \\
            1  &  2  &  5  &  0  &  5  &  5  &  5 \\
            4  &  4  &  3  &  4  &  0  &  6  &  6 \\
            5  &  5  &  5  &  5  &  5  &  0  &  7 \\
            6  &  6  &  6  &  6  &  6  &  6  &  0 \\       
        \end{bmatrix}
    \end{equation}\par
    ---------------- LOOP 3 ----------------\par
    距离矩阵
    \begin{equation}
        \begin{bmatrix}
            0  &  2  &  3  &  1  &  2  &  4  &  5 \\
            2  &  0  &  3  &  2  &  3  &  5  &  6 \\
            3  &  3  &  0  &  2  &  1  &  3  &  4 \\
            1  &  2  &  2  &  0  &  1  &  3  &  4 \\
            2  &  3  &  1  &  1  &  0  &  2  &  3 \\
            4  &  5  &  3  &  3  &  2  &  0  &  1 \\
            5  &  6  &  4  &  4  &  3  &  1  &  0 \\       
        \end{bmatrix}
    \end{equation}\par
    下一跳矩阵
    \begin{equation}
        \begin{bmatrix}
            0  &  2  &  4  &  4  &  4  &  4  &  4 \\
            1  &  0  &  3  &  4  &  4  &  4  &  4 \\
            5  &  2  &  0  &  5  &  5  &  5  &  5 \\
            1  &  2  &  5  &  0  &  5  &  5  &  5 \\
            4  &  4  &  3  &  4  &  0  &  6  &  6 \\
            5  &  5  &  5  &  5  &  5  &  0  &  7 \\
            6  &  6  &  6  &  6  &  6  &  6  &  0 \\       
        \end{bmatrix}
    \end{equation}\par
    此时迭代收敛\par
    以LOOP 2迭代中的a节点为例,a节点与b,c,d三个节点相邻,代价分别为2,5,1, a,b,c,d节点的距离矢量分别为
    \begin{equation}
        \begin{bmatrix}
            0  &  2  &  4  &  1  &  2  &  9 & \infty \\    
        \end{bmatrix}
    \end{equation} 
    \begin{equation}
        \begin{bmatrix}
            2  &  0  &  3  &  2  &  3  &  8 & \infty \\    
        \end{bmatrix}
    \end{equation}
    \begin{equation}
        \begin{bmatrix}
            3  &  3  &  0  &  2  &  1  &  3  &  6 \\    
        \end{bmatrix}
    \end{equation}
    \begin{equation}
        \begin{bmatrix}
            1  &  2  &  2  &  0  &  1  &  3 & \infty \\    
        \end{bmatrix}
    \end{equation}\par
    a向b,c,d广播自己的距离矢量,此次b,c,d的距离矢量未更新;\par
    b向a,c,d广播自己的距离矢量,四个节点的距离矢量未更新;\par
    c向a,b,d,f广播自己的距离矢量,a距离矢量的后两分量更新为9,11; b距离矢量的后两分量更新为6,9; d距离矢量最后1个分量更新为8(暂不考虑f的更新); \par
    d再广播距离矢量,a的距离矢量更新为
    \begin{equation}
        \begin{bmatrix}
            0  &  2  &  2+1  &  1  &  2  &  3+1  &  9+1 \\ 
        \end{bmatrix}
    \end{equation}\par
    下一步矢量的第3,6,7分量更新为4. 
    \item 打开实验文件“exp7 2.m”,阅读实验代码并运行实验文件。在本次实验中,我们在路由表收敛后,将路由节点6和7之间的链路断开(链路代价无穷大)。运行实验文件,观察matlab命令行输出的不同迭代轮次下的距离矩阵和下一跳矩阵。基于命令行的输出,观察并记录路由重新收敛的过程。结合D-V协议,分析该路由重新收敛的过程中的问题。\par
    记录如下:\par
    --------------- LOOP START --------------\par
    距离矢量矩阵
    \begin{equation}
        \begin{bmatrix}
            0  &  2   &  3   &  1   &  2   &  4  &  5 \\ 
            2  &  0   &  3   &  2   &  3    & 5  &  6 \\
            3   &  3   &  0 &    2   &  1  &   3  &  4 \\
            1   &  2   &  2  &   0   &  1   &  3  &  4 \\
            2   & 3   &  1   &  1   &  0   &  2  &  3 \\
            4   &  5   &  3   &  3   &  2   &  0 & \infty \\
          \infty  & \infty &  \infty  & \infty  & \infty  & \infty  &  0 \\       
        \end{bmatrix}
    \end{equation}
    下一跳矩阵
    \begin{equation}
        \begin{bmatrix}
            0  &  2  &  4  &  4  &  4  &  4  &  4 \\
            1  &  0  &  3  &  4  &  4  &  4  &  4 \\
            5  &  2  &  0  &  5  &  5  &  5  &  5 \\
            1  &  2  &  5  &  0  &  5  &  5  &  5 \\
            4  &  4  &  3  &  4  &  0  &  6  &  6 \\
            5  &  5  &  5  &  5  &  5  &  0  &  0 \\
            0  &  0  &  0  &  0  &  0  &  0  &  0 \\       
        \end{bmatrix}
    \end{equation}
    ---------------- LOOP 1 ----------------\par
    距离矩阵
    \begin{equation}
        \begin{bmatrix}
            0  &  2  &  3  &  1  &  2  &  4  &  5 \\
            2  &  0  &  3  &  2  &  3  &  5  &  6 \\
            3  &  3  &  0  &  2  &  1  &  3  &  4 \\
            1  &  2  &  2  &  0  &  1  &  3  &  4 \\
            2  &  3  &  1  &  1  &  0  &  2  &  7 \\
            4  &  5  &  3  &  3  &  2  &  0  &  5 \\
          \infty  &\infty  &\infty  &\infty  &\infty  &\infty  &  0 \\       
        \end{bmatrix}
    \end{equation}\par
    下一步矩阵
    \begin{equation}
        \begin{bmatrix}
            0  &  2  &  4  &  4  &  4  &  4  &  4 \\
            1  &  0  &  3  &  4  &  4  &  4  &  4 \\
            5  &  2  &  0  &  5  &  5  &  5  &  5 \\
            1  &  2  &  5  &  0  &  5  &  5  &  5 \\
            4  &  4  &  3  &  4  &  0  &  6  &  6 \\
            5  &  5  &  5  &  5  &  5  &  0  &  5 \\
            0  &  0  &  0  &  0  &  0  &  0  &  0 \\       
        \end{bmatrix}
    \end{equation}\par
    ---------------- LOOP 2 ----------------\par
    距离矩阵
    \begin{equation}
        \begin{bmatrix}
            0  &  2  &  3  &  1  &  2  &  4  &  5 \\
            2  &  0  &  3  &  2  &  3  &  5  &  6 \\
            3  &  3  &  0  &  2  &  1  &  3  &  6 \\
            1  &  2  &  2  &  0  &  1  &  3  &  6 \\
            2  &  3  &  1  &  1  &  0  &  2  &  5 \\
            4  &  5  &  3  &  3  &  2  &  0  &  7 \\
          \infty  &\infty  &\infty  &\infty  &\infty  &\infty  &  0 \\       
        \end{bmatrix}
    \end{equation}\par
    下一步矩阵
    \begin{equation}
        \begin{bmatrix}
            0  &  2  &  4  &  4  &  4  &  4  &  4 \\
            1  &  0  &  3  &  4  &  4  &  4  &  4 \\
            5  &  2  &  0  &  5  &  5  &  5  &  5 \\
            1  &  2  &  5  &  0  &  5  &  5  &  5 \\
            4  &  4  &  3  &  4  &  0  &  6  &  3 \\
            5  &  5  &  5  &  5  &  5  &  0  &  5 \\
            0  &  0  &  0  &  0  &  0  &  0  &  0 \\       
        \end{bmatrix}
    \end{equation}\par
    ---------------- LOOP 3 ----------------\par
    距离矩阵
    \begin{equation}
        \begin{bmatrix}
            0  &  2  &  3  &  1  &  2  &  4  &  7 \\
            2  &  0  &  3  &  2  &  3  &  5  &  8 \\
            3  &  3  &  0  &  2  &  1  &  3  &  8 \\
            1  &  2  &  2  &  0  &  1  &  3  &  8 \\
            2  &  3  &  1  &  1  &  0  &  2  &  7 \\
            4  &  5  &  3  &  3  &  2  &  0  &  9 \\
          \infty  &\infty  &\infty  &\infty  &\infty  &\infty  &  0 \\       
        \end{bmatrix}
    \end{equation}\par
    下一步矩阵
    \begin{equation}
        \begin{bmatrix}
            0  &  2  &  4  &  4  &  4  &  4  &  4 \\
            1  &  0  &  3  &  4  &  4  &  4  &  4 \\
            5  &  2  &  0  &  5  &  5  &  5  &  5 \\
            1  &  2  &  5  &  0  &  5  &  5  &  5 \\
            4  &  4  &  3  &  4  &  0  &  6  &  3 \\
            5  &  5  &  5  &  5  &  5  &  0  &  5 \\
            0  &  0  &  0  &  0  &  0  &  0  &  0 \\       
        \end{bmatrix}
    \end{equation}\par
    \dots\dots\par
    ---------------- LOOP 14 ----------------\par
    距离矩阵
    \begin{equation}
        \begin{bmatrix}
            0  &  2  &  3  &  1  &  2  &  4  & 29 \\
            2  &  0  &  3  &  2  &  3  &  5  & 30 \\
            3  &  3  &  0  &  2  &  1  &  3  & 30 \\
            1  &  2  &  2  &  0  &  1  &  3  & 30 \\
            2  &  3  &  1  &  1  &  0  &  2  & 29 \\
            4  &  5  &  3  &  3  &  2  &  0  & 31 \\
          \infty  &\infty  &\infty  &\infty  &\infty  &\infty  &  0 \\        
        \end{bmatrix}
    \end{equation}\par
    下一步矩阵
    \begin{equation}
        \begin{bmatrix}
            0  &  2  &  4  &  4  &  4  &  4  &  4 \\
            1  &  0  &  3  &  4  &  4  &  4  &  4 \\
            5  &  2  &  0  &  5  &  5  &  5  &  5 \\
            1  &  2  &  5  &  0  &  5  &  5  &  5 \\
            4  &  4  &  3  &  4  &  0  &  6  &  3 \\
            5  &  5  &  5  &  5  &  5  &  0  &  5 \\
            0  &  0  &  0  &  0  &  0  &  0  &  0 \\       
        \end{bmatrix}
    \end{equation}\par
    ---------------- LOOP 15 ----------------\par
    距离矩阵
    \begin{equation}
        \begin{bmatrix}
            0  &  2  &  3  &  1  &  2  &  4  & 31 \\
            2  &  0  &  3  &  2  &  3  &  5  & 32 \\
            3  &  3  &  0  &  2  &  1  &  3  & 32 \\
            1  &  2  &  2  &  0  &  1  &  3  & 32 \\
            2  &  3  &  1  &  1  &  0  &  2  & 31 \\
            4  &  5  &  3  &  3  &  2  &  0  & 33 \\
          \infty  &\infty  &\infty  &\infty  &\infty  &\infty  &  0 \\        
        \end{bmatrix}
    \end{equation}\par
    下一步矩阵
    \begin{equation}
        \begin{bmatrix}
            0  &  2  &  4  &  4  &  4  &  4  &  4 \\
            1  &  0  &  3  &  4  &  4  &  4  &  4 \\
            5  &  2  &  0  &  5  &  5  &  5  &  5 \\
            1  &  2  &  5  &  0  &  5  &  5  &  5 \\
            4  &  4  &  3  &  4  &  0  &  6  &  3 \\
            5  &  5  &  5  &  5  &  5  &  0  &  5 \\
            0  &  0  &  0  &  0  &  0  &  0  &  0 \\       
        \end{bmatrix}
    \end{equation}\par
    ---------------- LOOP 16 ----------------\par
    距离矩阵
    \begin{equation}
        \begin{bmatrix}
            0  &  2  &  3  &  1  &  2  &  4  & 33 \\
            2  &  0  &  3  &  2  &  3  &  5  & 34 \\
            3  &  3  &  0  &  2  &  1  &  3  & 34 \\
            1  &  2  &  2  &  0  &  1  &  3  & 34 \\
            2  &  3  &  1  &  1  &  0  &  2  & 33 \\
            4  &  5  &  3  &  3  &  2  &  0  & 35 \\
          \infty  &\infty  &\infty  &\infty  &\infty  &\infty  &  0 \\        
        \end{bmatrix}
    \end{equation}\par
    下一步矩阵
    \begin{equation}
        \begin{bmatrix}
            0  &  2  &  4  &  4  &  4  &  4  &  4 \\
            1  &  0  &  3  &  4  &  4  &  4  &  4 \\
            5  &  2  &  0  &  5  &  5  &  5  &  5 \\
            1  &  2  &  5  &  0  &  5  &  5  &  5 \\
            4  &  4  &  3  &  4  &  0  &  6  &  3 \\
            5  &  5  &  5  &  5  &  5  &  0  &  5 \\
            0  &  0  &  0  &  0  &  0  &  0  &  0 \\       
        \end{bmatrix}
    \end{equation}\par
    发现在迭代过程中,距离矩阵的最后一列的值不断增加,无法收敛;\par
    且节点g变得无法到达. 这是因为该算法中每个节点的距离矢量是基于临近节点的距离矢量以及与临近节点的距离进行更新,每个节点没有全局的信息,只有局域的信息,因此,在g节点不可达的时候,其余节点会根据已有的局域信息来更新其距离矢量与下一步矢量,使错误的信息在节点间不断更新迭代,而无法发现存在的无限循环问题。
\end{enumerate}
\subsection{OSPF协议与LS算法}
运行实验文件,观察Matlab命令行输出和生成的图片。观察并记录,链路状态协议下链路状态的泛洪过程。记录生成的图片中,“viewNode”路由节点的最小生成树随链路状态传播而变化的情况,结合Dijkstra算法,给出解释。\par
选择节点7, 记录如下:\par
完成第一次泛洪迭代,MST=
\begin{equation}
    \begin{bmatrix}
        0  &  0  &  0  &  0  &  0  &  0  &  0 \\
        0  &  0  &  0  &  0  &  0  &  0  &  0 \\
        0  &  0  &  0  &  0  &  0  &  5  &  0 \\
        0  &  0  &  0  &  0  &  0  &  0  &  0 \\
        0  &  0  &  0  &  0  &  0  &  2  &  0 \\
        0  &  0  &  5  &  0  &  2  &  0  &  1 \\
        0  &  0  &  0  &  0  &  0  &  1  &  0 \\   
    \end{bmatrix}
\end{equation}
\begin{figure}[htbp]
    \centering
    \includegraphics[scale=0.2]{pics/LS_1.jpg}
    \caption{第一次泛洪后的最小生成树}
\end{figure}
\par
完成第二次泛洪迭代,MST=
\begin{equation}
    \begin{bmatrix}
        0  &  0  &  5  &  0  &  0  &  0  &  0 \\
        0  &  0  &  3  &  0  &  0  &  0  &  0 \\
        5  &  3  &  0  &  0  &  1  &  0  &  0 \\
        0  &  0  &  0  &  0  &  1  &  0  &  0 \\
        0  &  0  &  1  &  1  &  0  &  2  &  0 \\
        0  &  0  &  0  &  0  &  2  &  0  &  1 \\
        0  &  0  &  0  &  0  &  0  &  1  &  0 \\   
    \end{bmatrix}
\end{equation}
\begin{figure}[htbp]
    \centering
    \includegraphics[scale=0.2]{pics/LS_2.jpg}
    \caption{第二次泛洪后的最小生成树}
\end{figure}
\par
完成第三次泛洪迭代,MST=
\begin{equation}
    \begin{bmatrix}
        0  &  0  &  0  &  1  &  0  &  0  &  0 \\
        0  &  0  &  0  &  2  &  0  &  0  &  0 \\
        0  &  0  &  0  &  0  &  1  &  0  &  0 \\
        1  &  2  &  0  &  0  &  1  &  0  &  0 \\
        0  &  0  &  1  &  1  &  0  &  2  &  0 \\
        0  &  0  &  0  &  0  &  2  &  0  &  1 \\
        0  &  0  &  0  &  0  &  0  &  1  &  0 \\   
    \end{bmatrix}
\end{equation}
\begin{figure}[htbp]
    \centering
    \includegraphics[scale=0.2]{pics/LS_3.jpg}
    \caption{第三次泛洪后的最小生成树}
\end{figure}
\par
完成第四次泛洪迭代,MST=
\begin{equation}
    \begin{bmatrix}
        0  &  0  &  0  &  1  &  0  &  0  &  0 \\
        0  &  0  &  0  &  2  &  0  &  0  &  0 \\
        0  &  0  &  0  &  0  &  1  &  0  &  0 \\
        1  &  2  &  0  &  0  &  1  &  0  &  0 \\
        0  &  0  &  1  &  1  &  0  &  2  &  0 \\
        0  &  0  &  0  &  0  &  2  &  0  &  1 \\
        0  &  0  &  0  &  0  &  0  &  1  &  0 \\   
    \end{bmatrix}
\end{equation}
\begin{figure}[htbp]
    \centering
    \includegraphics[scale=0.2]{pics/LS_4.jpg}
    \caption{第四次泛洪后的最小生成树}
\end{figure}\par
在 viewNode = 7 的实验中,节点7的最小生成树随着链路状态(LSA)的泛洪传播而逐步建立。初始阶段,节点7仅知道其直连邻居的链路状态,其计算出的最小生成树只包含直接相连的节点。随着泛洪的进行,其他节点的链路状态信息通过邻居逐跳传播到节点7,其链路状态数据库逐渐完善。每收到一轮新的链路状态信息,节点7都基于当前完整的全局网络拓扑视图,重新运行Dijkstra算法,计算出以自身为根到所有其他节点的最短路径树(SPT)。当所有节点的LSA都传播到全网后,节点7的SPT不再变化,路由收敛,此时生成的树是基于全网链路状态的最优路径集合。整个过程展示了OSPF中链路状态泛洪与Dijkstra算法结合,使每个节点能独立计算出无环、一致的最短路径。

\section{实验总结、体会和建议}
通过本次实验,我理解了距离矢量算法和链路状态算法在路由协议中的核心原理与实现差异。距离矢量算法通过邻居间迭代交换距离信息逐步收敛,但存在计数到无穷问题,收敛较慢。而链路状态算法通过全网泛洪链路状态通告,使每个节点获得全局拓扑后独立运行Dijkstra算法计算最短路径,收敛更快且路径更优。实验让我直观体会到网络拓扑结构对协议性能的影响,加深了对路由协议设计权衡的理解。
\end{document}