% Homework template for Inference and Information
% UPDATE: September 26, 2017 by Xiangxiang
\documentclass[a4paper]{article}
\usepackage{ctex}
\usepackage{amsmath, amssymb, amsthm}
\usepackage{moreenum}
\usepackage{mathtools}
\usepackage{url}
\usepackage{bm}
\usepackage{enumitem}
\usepackage{graphicx}
\usepackage{listings}
\usepackage{multirow}
\usepackage{siunitx}
\lstset{
    basicstyle          =   \sffamily,          % 基本代码风格
    keywordstyle        =   \bfseries,          % 关键字风格
    commentstyle        =   \rmfamily\itshape,  % 注释的风格,斜体
    stringstyle         =   \ttfamily,  % 字符串风格
    flexiblecolumns,                % 
    numbers             =   left,   % 行号的位置在左边
    showspaces          =   false,  % 是否显示空格,显示了有点乱,所以不显示了
    numberstyle         =   \zihao{-5}\ttfamily,    % 行号的样式,小五号,tt等宽字体
    showstringspaces    =   false,
    captionpos          =   t,      % 这段代码的名字所呈现的位置,t指的是top上面
    frame               =   lrtb,   % 显示边框
}

\lstdefinestyle{Python}{
    language        =   Python, % 语言选Python
    basicstyle      =   \zihao{-5}\ttfamily,
    numberstyle     =   \zihao{-5}\ttfamily,
    keywordstyle    =   \color{blue},
    keywordstyle    =   [2] \color{teal},
    stringstyle     =   \color{magenta},
    commentstyle    =   \color{red}\ttfamily,
    breaklines      =   true,   % 自动换行,建议不要写太长的行
    columns         =   fixed,  % 如果不加这一句,字间距就不固定,很丑,必须加
    basewidth       =   0.5em,
}
% \usepackage{subcaption}
\usepackage[caption=false,font=footnotesize,labelfont=rm,textfont=rm,subrefformat=parens]{subfig}
\usepackage{booktabs} % toprule
\usepackage[mathcal]{eucal}
\usepackage{color}
\usepackage{iidef}
\newif\ifans\anstrue
\newcommand{\myspace}[1]{\par\vspace{#1\baselineskip}}

\thecourseinstitute{\textnormal{通信与网络}}
\theterm{确定}
\begin{document}



\vspace{3mm}
\centerline{\textbf{\Large{实验3$\quad$载波传输实验报告}}}

\setcounter{section}{4}

\section{实验内容}
\subsection{BPSK调制}

\begin{enumerate}[label=(\arabic*)]
    \item 给出前10个符号的发送波形和接收波形图。如图1和图2所示。
    \begin{figure}[h]
        \centering
        \includegraphics[scale=0.45]{pics/1_1_1.png}
        \caption{前10个符号的发送波形图(BPSK)}
        \label{fig:BPSK1}
    \end{figure}
    \begin{figure}[h]
        \centering
        \includegraphics[scale=0.45]{pics/1_1_2.png}
        \caption{前10个符号的接收波形图(BPSK)}
        \label{fig:BPSK1}
    \end{figure}
    \item 画出前5个符号的匹配滤波输出波形(采样前),并在图中明确标出最佳采样时刻。如图3所示。
    \begin{figure}[h]
        \centering
        \includegraphics[scale=0.45]{pics/1_2.png}
        \caption{前5个符号的匹配滤波输出波形(BPSK)}
        \label{fig:BPSK2}
    \end{figure}
	\item 分别比较误符号率和误比特率的理论值与仿真值。如图4和图5所示。
    \begin{figure}[h]
        \centering
        \includegraphics[scale=0.45]{pics/1_3_1.png}
        \caption{误符号率的理论值与仿真值(BPSK)}
        \label{fig:BPSK3_1}
    \end{figure}
    \begin{figure}[h]
        \centering
        \includegraphics[scale=0.45]{pics/1_3_2.png}
        \caption{误比特率的理论值与仿真值(BPSK)}
        \label{fig:BPSK3_2}
    \end{figure}
	\item 分析误码率随采样偏差变化的规律。如图6所示。
    \begin{figure}[h]
        \centering
        \includegraphics[scale=0.45]{pics/1_4.png}
        \caption{误码率随采样偏差变化的规律(BPSK)}
        \label{fig:BPSK4}
    \end{figure}
	\item 比较画出的功率谱密度与理论功率谱密度。如图7所示。
    \begin{figure}[h]
        \centering
        \includegraphics[scale=0.45]{pics/1_5.png}
        \caption{功率谱密度与理论功率谱密度(BPSK)}
        \label{fig:BPSK5}
    \end{figure}


\end{enumerate}


\subsection{4PAM调制}

\begin{enumerate}[label=(\arabic*)]
    \item 给出前10个符号的发送波形和接收波形图。如图8和图9所示。
    \begin{figure}[h]
        \centering
        \includegraphics[scale=0.45]{pics/2_1_1.png}
        \caption{前10个符号的发送波形图(4PAM)}
        \label{fig:4PAM1_1}
    \end{figure}
    \begin{figure}[h]
        \centering
        \includegraphics[scale=0.45]{pics/2_1_2.png}
        \caption{前10个符号的接收波形图(4PAM)}
        \label{fig:4PAM1_2}
    \end{figure}
    \item 画出前5个符号的匹配滤波输出波形(采样前),并在图中明确标出最佳采样时刻。如图10所示。
    \begin{figure}[h]
        \centering
        \includegraphics[scale=0.45]{pics/2_2.png}
        \caption{前5个符号的匹配滤波输出波形(4PAM)}
        \label{fig:4PAM2}
    \end{figure}
	\item 分别比较误符号率和误比特率的理论值与仿真值。如图11和图12所示。
    \begin{figure}[h]
        \centering
        \includegraphics[scale=0.45]{pics/2_3_1.png}
        \caption{误符号率的理论值与仿真值(4PAM)}
        \label{fig:4PAM3_1}
    \end{figure}
    \begin{figure}[h]
        \centering
        \includegraphics[scale=0.45]{pics/2_3_2.png}
        \caption{误比特率的理论值与仿真值(4PAM)}
        \label{fig:4PAM3_2}
    \end{figure}
	\item 分析误码率随采样偏差变化的规律。如图13所示。
    \begin{figure}[h]
        \centering
        \includegraphics[scale=0.45]{pics/2_4.png}
        \caption{误码率随采样偏差变化的规律(4PAM)}
        \label{fig:4PAM4}
    \end{figure}
	\item 比较画出的功率谱密度与理论功率谱密度。如图14所示。
    \begin{figure}[h]
        \centering
        \includegraphics[scale=0.45]{pics/2_5.png}
        \caption{功率谱密度与理论功率谱密度(4PAM)}
        \label{fig:4PAM5}
    \end{figure}

\end{enumerate}

\subsection{对比BPSK和4PAM的误码曲线}
\begin{enumerate}[label=(\arabic*)]
    \item 同样的$\frac{E_b}{N_0}$,BPSK的误码率明显优于4PAM。\par
    这是因为,BPSK每个符号只携带1 bit,符号距离大;4PAM 每个符号携带2 bit,但其4个电平之间距离较小。在相同比特能量$E_b$下:
    \[d_{min,BPSK}=d_{min,4PAM}=2A\]
    但4PAM的归一化幅度为$-3A,-A,A,3A$,等效符号距离较小,因此抗噪能力更差。\par
    这与调制理论一致:多电平调制、频谱效率提高,但抗噪性能下降。

    \item SER(误符号率)差距比BER更明显。\par
    由于4PAM每符号携带 2bit:1 bit错误$\neq$1 symbol错误,Gray coding只能减少bit-error,但无法改变symbol-error。

    \item 定时偏差对4PAM的影响更严重。\par
    因为BPSK只有两个电平,判决门限简单(0 点)。\par
    但4PAM有四个电平,判决门限有多个点(-2A, 0, 2A),因此采样偏差越大,接收波形畸变越明显,误判概率显著增加,4PAM的“误码率随采样偏差变化曲线”更陡。
\end{enumerate}

\section{实验总结、体会和建议}
本次实验通过搭建 BPSK 与 4PAM 的完整通信链路,让我更直观地理解了数字调制系统的实现过程和性能差异。实验结果显示,BPSK 由于符号间距大、判决简单,误码性能明显优于 4PAM,而 4PAM 虽然频谱效率更高,但对噪声和采样偏差更敏感,误码性能下降更快。这些现象与理论完全一致,也让我体会到调制阶数提高所带来的性能权衡。此外,匹配滤波前后信号的对比帮助我理解了匹配滤波提升信噪比的作用。通过本次实验,我对调制方式选择、同步精度要求以及噪声建模等关键问题有了更清晰的认识。

\end{document}




%%% Local Variables:
%%% mode: late\rvx
%%% TeX-master: t
%%% End:
