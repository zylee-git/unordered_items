% Homework template for Inference and Information
% UPDATE: September 26, 2017 by Xiangxiang
\documentclass[a4paper]{article}
\usepackage{ctex}
\usepackage{amsmath, amssymb, amsthm}
\usepackage{moreenum}
\usepackage{mathtools}
\usepackage{url}
\usepackage{bm}
\usepackage{enumitem}
\usepackage{graphicx}
\usepackage{listings}
\usepackage{multirow}
\usepackage{siunitx}
\lstset{
    basicstyle          =   \sffamily,          % 基本代码风格
    keywordstyle        =   \bfseries,          % 关键字风格
    commentstyle        =   \rmfamily\itshape,  % 注释的风格,斜体
    stringstyle         =   \ttfamily,  % 字符串风格
    flexiblecolumns,                % 
    numbers             =   left,   % 行号的位置在左边
    showspaces          =   false,  % 是否显示空格,显示了有点乱,所以不显示了
    numberstyle         =   \zihao{-5}\ttfamily,    % 行号的样式,小五号,tt等宽字体
    showstringspaces    =   false,
    captionpos          =   t,      % 这段代码的名字所呈现的位置,t指的是top上面
    frame               =   lrtb,   % 显示边框
}

\lstdefinestyle{Python}{
    language        =   Python, % 语言选Python
    basicstyle      =   \zihao{-5}\ttfamily,
    numberstyle     =   \zihao{-5}\ttfamily,
    keywordstyle    =   \color{blue},
    keywordstyle    =   [2] \color{teal},
    stringstyle     =   \color{magenta},
    commentstyle    =   \color{red}\ttfamily,
    breaklines      =   true,   % 自动换行,建议不要写太长的行
    columns         =   fixed,  % 如果不加这一句,字间距就不固定,很丑,必须加
    basewidth       =   0.5em,
}
% \usepackage{subcaption}
\usepackage[caption=false,font=footnotesize,labelfont=rm,textfont=rm,subrefformat=parens]{subfig}
\usepackage{booktabs} % toprule
\usepackage[mathcal]{eucal}
\usepackage{color}
\usepackage{iidef}
\newif\ifans\anstrue
\newcommand{\myspace}[1]{\par\vspace{#1\baselineskip}}

\thecourseinstitute{\textnormal{通信与网络}}
\theterm{确定}
\begin{document}



\vspace{3mm}
\centerline{\textbf{\Large{实验5$\quad$频分多址实验报告}}}

\setcounter{section}{4}

\section{实验内容}
% \subsection{BPSK调制}

\begin{enumerate}[label=(\arabic*)]
    \item 绘制发射信号波形功率谱。根据功率谱密度计算各用户的功率和总功率。\par
    发射信号功率谱密度如图 \ref{fig:1}所示。
    \begin{figure}[htbp]
        \centering
        \includegraphics[scale=0.25]{pics/1.png}
        \caption{发射信号波形功率谱}
        \label{fig:1}
    \end{figure}\par
    总功率:$P_{all}=\int_{-\infty}^{\infty}S_xdx=400$。\par
    各用户功率:\par
    $P_1=2\times\int_{f_{lc}+\frac{\Delta f_c}{2}}^{f_{lc}-\frac{\Delta f_c}{2}}S_Xdx=91.7304$\par
    $P_2=2\times\int_{f_{lc}+\frac{3\Delta f_c}{2}}^{f_{lc}+\frac{\Delta f_c}{2}}S_Xdx=98.1359$\par
    $P_3=2\times\int_{f_{lc}+\frac{5\Delta f_c}{2}}^{f_{lc}+\frac{3\Delta f_c}{2}}S_Xdx=98.6723$\par
    $P_4=2\times\int_{f_{lc}+\frac{7\Delta f_c}{2}}^{f_{lc}+\frac{5\Delta f_c}{2}}S_Xdx=96.1617$.\newline
    作为近似,认为各用户的功率都在$f\in[f_c-\frac{\Delta f_c},{f_c+\Delta f_c}]$之间,积分用有限项求和替代。各个用户功率的理论值为100,这种根据功率谱密度计算的方法与理论值存在一定偏差,原因是计算时采取的近似方法带来了误差。
    \item 在多个噪声功率谱密度下,比较发送信号和接收信号,统计误符号率、误比特率。画出误比特率与$E_b/n_0$之间的关系曲线,并与理论误比特率曲线进行对比。
    选取几个有代表性的噪声功率谱密度,绘制前四个符号发送波形与接收波形图如图 \ref{fig:2}, 图\ref{fig:3}, 图\ref{fig:4}与图\ref{fig:5}. 发现波形的失真程度随$n_0$增大而增大。
    \begin{figure}[htbp]
        \centering
        \includegraphics[scale=0.25]{pics/2.png}
        \caption{$n_0=0.1$时发送波形与接收波形}
        \label{fig:2}
    \end{figure}
    \begin{figure}[htbp]
        \centering
        \includegraphics[scale=0.25]{pics/3.png}
        \caption{$n_0=0.5$时发送波形与接收波形}
        \label{fig:3}
    \end{figure}
    \begin{figure}[htbp]
        \centering
        \includegraphics[scale=0.25]{pics/3.png}
        \caption{$n_0=1$时发送波形与接收波形}
        \label{fig:4}
    \end{figure}
    \begin{figure}[htbp]
        \centering
        \includegraphics[scale=0.25]{pics/4.png}
        \caption{$n_0=5$时发送波形与接收波形}
        \label{fig:5}
    \end{figure}\par
    选取几个有代表性的$n_0$,统计相应条件下的误比特率,并计算相应理论值,如表 \ref{tab:1}.
    \begin{table}[htbp]
        \centering
        \begin{tabular}{|c|c|c|c|c|c|}
            \hline
            $n_0$ & 0.1 & 0.5 & 1 & 5 \\
            \hline
            $\frac{E_b}{n_0}$ & 10 & 2 & 1 & 0.2 \\
            \hline
            用户1BER & $7\times10^{-6}$ & $0.0232$ & $0.0789$ & $0.2638$ \\
            \hline
            用户2BER & $7\times10^{-6}$ & $0.0230$ & $0.0800$ & $0.2665$ \\
            \hline
            用户3BER & $2\times10^{-6}$ & $0.0232$ & $0.0792$ & $0.2631$ \\
            \hline
            用户4BER & $4\times10^{-6}$ & $0.0231$ & $0.0791$ & $0.2638$ \\
            \hline
            理论值 & $3.8721\times10^{-6}$ & $0.0228$ & $0.0786$ & $0.2635$ \\
            \hline
        \end{tabular}
        \label{tab:1}
    \end{table}\par
    误比特率与$E_b/n_0$之间的关系曲线见图 \ref{fig:6}. 可以看出实际误比特率与理论值基本上相差极小,说明频分多址没有引入附加的误码率。
    \begin{figure}[htbp]
        \centering
        \includegraphics[scale=0.25]{pics/6.png}
        \caption{误比特率与$E_b/n_0$之间的关系曲线}
        \label{fig:6}
    \end{figure}

\end{enumerate}





\section{选做题}

\begin{enumerate}[label=(\arabic*)]
    \item 重复必做实验内容的(1)-(2)
    \item 对于该选做题中的现象进行分析,并尝试从理论上给出解释。
\end{enumerate}

\section{实验总结、体会和建议}
本次实验通过MATLAB仿真了FDMA系统中的BPSK调制与解调过程,观察了多载波传输的信号特性,统计了误比特率并与理论值进行了对比分析。实验还绘制了发射信号的功率谱密度图像,进一步理解了频分多址的工作机制与正交载波在系统中的重要性。通过本实验,我们巩固了通信原理中多址接入、功率谱等基础概念,提升了仿真与分析能力。
\end{document}




%%% Local Variables:
%%% mode: late\rvx
%%% TeX-master: t
%%% End:
